En el presente Trabajo Práctico diseñamos e implementamos una base de datos no relacional para guardar el histórico de los
Campeonatos Mundiales de Taekwon-do ITF. Esta base de datos permite guardar los enfrentamientos, con su respectivo
resultado, de las categorías que conforman cada campeonato. Para realizar esta labor construimos un Modelo Conceptual representado con un
Diagrama de Entidad Relación(DER) que sirvió como base para el Diagrama de Interacción de Documentos(DID). Este último está
pensado y construido para optimizar las consultas pedidas. La implementación fue realizada usando la base de datos basada
en documentos \textbf{RethinkDb}. A lo largo del presente informe detallaremos las distintas partes del DER, las decisiones
tomadas para construir el DID y mostraremos la documentación del diseño lógico de la base de datos(usando \textbf{JSONSchema}).

\subsection{Asunciones}
\begin{enumerate}
\item Consideramos que no se debe modelar la totalidad del problema propuesto en el TP1. Solamente representamos y trabajamos
sobre la información dada en el enunciado y las consultas pedidas. En este contexto, omitimos muchas entidades que no mostraban
participación alguna en el modelo pedido. Por ejemplo, la entidad Pais.
\item Asumimos que, en un mismo año, no hay más de un campeonato.
\item Obviamos los enfrentamientos por equipos y consideramos que en un combate determinado de cualquier categoria solo hay
dos participantes. Por lo tanto, solo hay un vencedor por enfrentamiento.
\end{enumerate}
