En esta sección mostraremos el Diagrama de Interrelación de Documentos (DID) creado a partir del DER mostrado en la sección
anterior.

\begin{figure}[H]
  \centering
    \includegraphics[scale=0.4]{imagenes/DID.png}
  \caption{Diagrama de Interrelación de Documentos}
\end{figure}

A continuación mostraremos la forma en la que se resolvieron las consultas pedidas y justificaremos las decisiones
tomadas para optimizarlas.

\subsection{Consultas}
\begin{enumerate}
  \item Se pide la cantidad de enfrentamientos ganados por competidor para un campeonato dado. Se nos da como parámetro
  el año del campeonato. Para resolver la consulta, recorremos todos los enfrentamientos. Contamos la cantidad de
  enfrentamientos ganados para cada competidor que coincidan con el año del campeonato. Esto lo podemos realizar de forma óptima
  ya que el enfrentamiento referencia a campeonato, por lo tanto tiene su año. Además, referencia al ganador y al
  perdedor, por lo que puede usarlos para contar la cantidad de victorias por cada competidor.

  \item Se pide la cantidad de medallas por nombre de escuela en toda la historia. Iteramos en el documento Escuela y por cada una
  recorremos la ListaCompetidores. Para cada competidor, realizamos la suma de las medallas de los tres tipos (oro, plata y bronce).
  Esta sumatoria se beneficia del hecho de que cada escuela posee la ListaCompetidores gracias a que se tomo la decisión de embeber parcialmente a los competidores en el documento Escuela. La lista de competidores contiene la cantidad de medallas de cada uno de los tres tipos que cada competidor obtuvo
  históricamente. Solo hace falta sumar los tres tipos para obtener el histórico de medallas del competidor.

  \item Para cada escuela, se pide el campeonato donde ganó más medalla. La consulta se divide en dos partes. La primer parte consiste
  en recorrer todos los campeonatos y guardarnos las categorías que están en ListaCategorías, utilizando para esto el año del
  campeonato. En la segunda parte recorremos todas las entradas en el documento Escuela. Por cada una de estas, utilizamos las
  categorías guardadas en el paso anterior. Recorremos todas las categorías y, por cada una, contamos la cantidad de ganadores de
  medallas que pertenecen a la escuela.
  
  Esta consulta está optimizada usando que Campeonato tiene embebido a Categoría, con lo
  cual podemos adquirir de cada campeonato los ganadores de medallas de los tres tipos. Además Categoría tiene embebido parcialmente
  a Competidor por el atributo NombreEscuela, con lo cual podemos usar este dato para identificar a la escuela que pertenece cada
  medalla. Por último, el documento Escuela tiene una referencia a Campeonato. Esto significa que Escuela tiene los yearCampeonato
  de todos en los que mando competidores.
  
  En el primer paso habíamos guardado las categorías por el año del campeonato y,
  gracias a esta referencia de Campeonato en Escuela, podemos acceder a las categorías. Por consiguiente, tenemos acceso a la
  información sobre los ganadores de medallas. Dado que necesitamos la información de las medallas, la consulta pareciera que
  se puede resolver tanto usando el documento Campeonato como el documento Competidor. Sin embargo, esto no es cierto ya que
  a través de competidor no podemos diferenciar los campeonatos en los cuales cada uno gano las medallas. Por eso necesitamos recorrer
  los campeonatos para obtener las categorías y después ver en cuales los competidores de cada escuela resultaron vencedores.

  \item Se pide los árbitros que participaron en al menos 4 campeonatos. La consulta se realiza recorriendo los documentos
  de Árbitro contando los Campeonatos usando ListaCampeonatos. Finalmente nos quedamos con los árbitros donde la ListaCampeonatos
  tiene al menos 4 campeonatos distintos. Para cada árbitro, el proceso se resume en contar los campeonatos en la lista.
  Podemos realizar esto gracias a ListaCampeonatos, la cual obtenemos gracias a que el documento Arbitro posee referencias
  al documento Campeonato.

  \item Se pide las escuelas que han presentado el mayor número de competidores en cada campeonato. Recorremos el conjunto de documentos
  Campeonato y para cada uno recorremos su ListaCompetidores y cuentamos, por cada NombreEscuela, la cantidad de competidores.
  La consulta es optimizada debido a que los Campeonatos tienen embebido parcialmente a Competidor por el atributo
  NombreEscuela. Con lo cual, no se necesita acceder a otro documento que no sean todas las entradas de Campeonatos.
  Esto último es necesario ya que necesitamos información de todos los Campeonatos para encontrar a las escuelas en las
  que más competidores participaron.

  \item Obtener los competidores que más medallas obtuvieron por modalidad. Por cada entrada del documento Campeonato,
  contamos las medallas de cada tipo que coinciden con cada modalidad. Después nos quedamos con el DNICompetidor a los cuales
  son ganadores de más medallas de cada tipo. Esta consulta se beneficia de que el Campeonato tiene embebido
  el documento Categoría. Por lo tanto, no es necesario acceder a ningún otro documento que no sea Campeonato. Desde el embebido
  de Categoría obtenemos los DNI's de los competidores que resultaron ganadores.

\end{enumerate}
